\chapter{Struktur, ledelse og organisasjon}
\section{Hovedstyret}
Hovedstyret er linjeforeningens høyeste organ mellom generalforsamlingene. Hovedstyrets medlemmer velges på generalforsamlingen og skal drive linjeforeningen mellom generalforsamlingene. For at Hovedstyret skal være beslutningsdyktig må minst fire representanter være tilstede. \newline

Ingen kan inneha to verv i Hovedstyret. Leder har dobbeltstemme ved stemmelikhet. Hovedstyrets møter er lukket, men gjester kan inviteres dersom Hovedstyret ønsker \linebreak dette. Leder av Hovedstyret skal avholde medarbeidersamtaler for hovedstyre-\linebreak medlemmer minst en gang i året. Hovedstyret skrives med stor 'H' på samme måte som egennavn.

%----
\under{Hovedstyrets sammensetning}{
Hovedstyret består av:
\begin{liste}
	\item Leder
	\item Nestleder
	\item Leder av Arrangementskomiteen
	\item Leder av Bank- og økonomikomiteen
	\item Leder av Bedriftskomiteen
	\item Leder av Drifts- og utviklingskomiteen
	\item Leder av Fag- og kurskomiteen
	\item Leder av Profil- og aviskomiteen
	\item Leder av Trivselskomiteen
\end{liste}
}

%----
\under{Krav til kandidater}{
En kandidat må ha innehatt et verv i linjeforeningen i minst ett semester med unntak av kandidatur til stillingen som leder av bank- og økonomikomiteen.
}

%----
\under{Hovedstyrets virke}{

\begin{liste}
	\item Hovedstyret fører logg over linjeforeningens aktiviteter og fremlegger årsberetning for generalforsamlingen.
	\item Hovedstyret skal formidle saker og vedtak til komitemedlemmene via den respektive komitelederen.
	\item Representantene av Hovedstyret kan uttale seg og handle på linjeforeningens vegne når vedkommendes sak er avklart i Hovedstyret. Ved uklarheter skal leder uttale seg.
	\item Hovedstyret har fullmakt til å fylle eventuelle tomme verv ved behov.
	\item Hovedstyret er ansvarlig for komiteopptak og skal føre intervju- og søknadsprosesser. 			Hovedstyret skal godkjenne komiteenes retningslinjer på første hovedstyremøte etter endt generalforsamling.
\end{liste}
}



%----
\vspace{-23pt}
\section{Komiteer}
\label{sec:komiteer}

Alle komiteer består av minimum en leder og en økonomiansvarlig, hvorav det er valgfritt om økonomiansvarlig er Leder av banKom eller en som sitter i den \linebreak respektive komiteen. Økonomiansvarlig skal holde orden på økonomien \mbox{i tråd med §\ref{chap:okonomi}}. Enhver komite skal utarbeide sine egne retningslinjer som skal legges frem for og godkjennes av Hovedstyret. \newline

Leder av en komite har mulighet til å opprette egne komitestillinger, for eksempel nestleder, og det er opp til komiteen selv å fylle stillingen. Enhver leder skal avholde medarbeidersamtaler minst en gang i året. \newline

Bare medlemmer av linjeforeningen kan inneha verv og disse opphører dersom studenten bytter til studium som ikke kvalifiserer til medlemskap i linjeforeningen \mbox{etter §\ref{chap:medlemskap}.} \newline

%----
\vspace{-10pt}
\under{Arrangementskomiteen}{
Komiteens hovedoppgave er å koordinere og gjennomføre sosiale arrangement. Komiteens navn forkortes arrKom.
}

%----
\vspace{-10pt}
\under{Bank- og økonomikomiteen}{
Komiteens hovedoppgave er administrere linjeforeningens økonomi. Komiteens  medlemmer utgjøres av de økonomiansvarlige fra de andre komiteene. Komiteens navn forkortes banKom.
}

%----
\vspace{-10pt}
\under{Bedriftskomiteen}{
Komiteens hovedoppgaver er å være et bindeledd mellom linjeforeningens medlemmer og næringslivet, og å utarbeide en hovedsamarbeidssamtale med en bedrift for Linjeforeningen Online. Komiteens navn forkortes bedKom.
}

%----
\vspace{-10pt}
\under{Drifts- og utviklingskomiteen}{
Komiteens hovedoppgave er å utvikle og vedlikeholde linjeforeningens datasystemer. Komiteens navn forkortes dotKom.
}

%----
\vspace{-10pt}
\under{Fag- og kurskomiteen}{
Komiteens hovedoppgave er å koordinere og gjennomføre arrangement som tilbyr faglig innhold, primært for linjeforeningens egne medlemmer. Komiteens navn forkortes fagKom.
}

%----
\vspace{-10pt}
\under{Pensjonistkomiteen}{
Medlemskap i pensjonistkomiteen kan søkes til etter avsluttet studie på informatikk ved NTNU.\newline

Et medlemsskap i pensjonistkomiteen varer livet ut med mindre man på nytt blir medlem av en av linjeforeningens komiteer; man vil da få permisjon fra pensjonistkomiteen i den tidsperioden man er med i
den andre komiteen. Alle tidligere informatikkstudenter har anledning til å søke om medlemskap i pensjonistkomiteen.\newline

Det påpekes at epostlistene til pensjonistkomiteen ikke burde benytte seg av epostadresser fra universitetet, det bør heller benyttes en permanent, privat epostadresse.\newline

Komiteens navn forkortes pangKom.
}

%----
\vspace{-10pt}
\under{Profil- og aviskomiteen}{
Komiteens hovedoppgave er å sikre kvalitet på profileringsmateriell, samt gi ut
linjeforeningens tidsskrift. Komiteens navn forkortes proKom.
}

%----
\vspace{-10pt}
\under{Trivselskomiteen}{
Komiteens hovedoppgave er å sørge for økt trivsel blant informatikere i hverdagen. Komiteens sekundære oppgave er å ha ansvaret for linjeforeningskontoret. Ansvaret for kontoret innebærer å planlegge og følge opp kontorvakter, tilrettelegge for møteaktivitet og sørge for at komitemedlemmer har tilgang til kontoret. Komiteens navn forkortes triKom.
}

%----
\vspace{-10pt}
\under{Seniorkomiteen}{
Komiteens hovedoppgave vil være å bistå med kunnskap, erfaring og assistanse i Linjeforeningens daglige drift. For å søke seg til Seniorkomiteen må man ha hatt et aktivt verv i Linjeforeningen i minst fire semester. Seniorkomiteen tar selv opp medlemmer, men medlemmene må godkjennes av Hovedstyret. Dersom Seniorkomiteen ikke har mulighet til å gjennomføre opptaket vil Hovedstyret utføre denne oppgaven. \newline

Seniorkomiteen velger selv sin leder. Leder av Seniorkomiteen har møte- og talerett i Hovedstyret.
}

%----
\vspace{-10pt}
\section{Nodekomiteer}{
En nodekomite er underlagt en av komiteene beskrevet i §4.2. Komiteen som nodekomiteen er underlagt kalles heretter forelderkomiteen.

Forelderkomiteen nedsetter en leder for nodekomiteen ved opprettelse. Ved tillatelse fra forelderkomiteen kan nodekomiteen i stedet velge sin egen leder etter at medlemmer av nodekomiteen er tatt opp. Denne lederen er ansvarlig for å ta opp nødvendige medlemmer, opprette nødvendige stillinger og holde forelderkomiteen løpende underrettet om komiteens status. Leder for forelderkomiteen er i sin tur ansvarlig for å holde Hovedstyret løpende underrettet om nodekomiteens status.
}

%----
\under{Periodiske komiteer}{
Dette er nodekomiteer som eksisterer i kortere perioder, men med faste intervall, for å ta seg av spesielle begivenheter. Retningslinjer for periodiske komiteer skal fremlegges for, og godkjennes av, Hovedstyret snarlig etter opprettelse av komiteen i stedet for etter generalforsamlingen.
}

%----
\underunder{Jubileumskomiteen}{
Komiteens hovedoppgave er å organisere arrangement i forbindelse med linjeforeningens jubileer. Komiteens forelderkomite er arrKom. Komiteens navn forkortes jubKom. \newline

Hovedstyret skal spare kr. 10.000,- hvert år til neste jubileum. Beløpet kan justeres av Hovedstyret dersom de ser behov for det. Hovedstyret kan etter behov vedta å ikke spare noe et år dersom linjeforeningen har havnet i en vanskelig økonomisk situasjon. Pengene skal settes over på en egen konto og blir da øremerket til gjennomføring av linjeforeningens neste jubileum.
}

%----
\underunder{Velkomstkomiteen}{
Komiteens hovedoppgave er å organisere fadderperiode for nye studenter som oppfyller de krav for medlemskap som er listet under §\ref{chap:medlemskap}. Komiteens navn forkortes velKom.
}

\underunder{Ekskursjonskomiteen}{Komiteens hovedoppgave er å organisere ekskursjon for studentene i 3.klasse. Komiteen kan selv velge å tillate studenter i eldre årskull å bli med på ekskursjonen, dersom studenten det gjelder ikke har vært på ekskursjon i regi av linjeforeningen tidligere. Ekskursjonskomiteen skal også jobbe for å sikre økonomisk støtte til ekskursjonen. Komiteens forelderkomite er fagKom. Komiteens navn forkortes eksKom.}

%----
\under{Krysskomiteer}{
Dette er nodekomiteer som har medlemmer utenfor linjeforeningen. Krysskomiteer kan operere som frittstående organisasjoner, men vil likevel få tilgang til tjenestene linjeforeningens komiteer tilbyr. Det bemerkes at det som beskrives her kun gjelder komitemedlemmer i en krysskomite som også er medlemmer av linjeforeningen. Krysskomiteer kan selv bestemme om de ønsker å ha retningslinjer slik som de andre komiteene.
}

%----
\underunder{Casual Gaming}{
Komiteens hovedoppgave er å organisere LAN. Komiteens forelderkomite er arrKom.
}

%----
\under{Redaksjonen}{
Komiteens hovedoppgave er å gi ut linjeforeningens avis. Redaktøren står fritt fra linjeforeningen, men er underlagt de retningslinjer og avtaler som finnes mellom linjeforeningen og forelderkomiteen. Tidsskriftet skal, følge Vær Varsom-plakaten definert av Pressens Faglige Utvalg. Redaktøren følger på lik måte Redaktørplakaten definert av Pressens Faglige Utvalg. Redaktøren står fritt til å velge redaksjonsmedlemmer, også blant personer utenfor linjeforeningen. Merk at personer som ikke innfrir krav til medlemskap som definert under §5 ikke får medlemskap i linjeforeningen utenfor redaksjonen, men kan fritt inkluderes på interne arrangementer for medlemmer med verv. Komiteens forelderkomite er proKom.
}

\under{Informatikernes IT-ekskursjon}{Komiteens hovedoppgave er å arrangere ekskursjon til Oslo for masterstudenter. Denne arrangeres hver høst og har som hovedformål å gjøre masterstudentene kjent med hvilke muligheter som finnes etter endt studie. Komiteen er selvstyrt og tar alle avgjørelser angående ekskursjonen. Komiteens forelderkomite er bedKom. Komiteens navn forkortes itex}

\under{Applikasjonskomiteen}{Komiteens hovedoppgave er å utvikle og drifte nettleserutvidelser, mobilapplikasjoner, infoskjermer og tilhørende tjenester som f.eks. mellomlagstjenester og mikrokontrollere. Komiteens navn forkortes appKom. Komiteens forelderkomite er dotKom. Komiteen skal ikke drive med oppgaver som overlapper med dotKoms oppgaver. Komiteen skal være nært tilknyttet dotKom gjennom komiteens leder. Leder av appKom har plikt til å rapportere til dotKom om komiteens drift, og skal ikke pålegges andre arbeidsoppgaver i dotKom. Per §4.3 velges komiteleder for appKom internt i dotKom. Ny leder velges snarest etter linjeforeningens ordinære generalforsamling. Dersom dotKom enten ikke finner en leder eller ikke ønsker å finne en leder, skal appKom selv velge en leder blant sittende komitemedlemmer.
Komiteen har løpende åpent opptak, åpen dør og lav terskel for deltakelse. Enhver Onliner som oppfyller kravene til medlemskap i §5 skal kunne bli med på møter. Ettersom det vil være enkelt å delta på møter stilles det to formelle krav før komiteleder gir tilbud om å bli komitemedlem. Kravene er deltakelse på fire påfølgende møter og fullførelse av en opptaksprøve som bestemmes internt i komiteen.
På grunn av utfordringer som kan oppstå med åpent opptak gis komiteleder fullmakt til å begrense adgang til møtene for deltakere som ikke er medlemmer av komiteen. Det spesifiseres at begrensning i hovedsak er ment å brukes dersom komiteen er for stor, eller dersom en deltaker har uakseptabel oppførsel eller innsats. Dersom et komitemedlem har en periode med inaktivitet eller lav deltakelse kan leder velge å avslutte komitevervet til vedkommende.}

\under{Andre nodekomiteer}{Andre nodekomiteer kan opprettes av onlinemedlemmer som ønsker å dekke et behov som gagner informatikkstudenter. Disse komitéene formulerer sine egne retningslinjer og budsjett som deretter godkjennes av Hovedstyret. Hovedstyret avgjør også hvorvidt komitéen underlegges en eksisterende komité, eller Hovedstyret selv. Nodekomitéer som har eksitert i to år eller mer skal ha egen budsjett post i foreldrekomitéen og kan bare offisielt nedlegges av Generalforsamlingen.}

\section{Permisjon eller oppsigelse fra komite}{

Ved permisjon fra en komite er man fullstendig fritatt de pliktene komitevervet medførte. Man er i tillegg unntatt visse administrative rettigheter som bestemmes av den respektive komiteen. \newline

Ved oppsigelse fra en komite mister man de rettigheter man har tilegnet seg internt i komiteen. Man har likevel rett til å bruke de av linjeforeningens og komiteens effekter man har fortjent, samt å bli med i pangKom dersom man møter kravene for opptak.
}

%----
\under{Pause i sitt engasjement}{
Et komitemedlem kan søke om permisjon fra den respektive komite når medlemmet ønsker å ta en pause fra komiteen. Man må ha \mbox{sittet} i en komite i minst ett semester for å kunne søke permisjon. Dersom \linebreak permisjonen varer lengre enn to semestere vil medlemmets verv opphøre. Permisjonslengde kan ikke overstige komitemedlemmets fartstid i den \linebreak respektive komiteen.
}

\under{Verv i Hovedstyret}{
Dersom et komitemedlem blir valgt til et av følgende hovedstyreverv vil medlemmet automatisk få permisjon fra sin komite, og kan fritt \linebreak returnere til denne ved endt engasjement i Hovedstyret:
\begin{liste}
	\item Leder
	\item Nestleder
	\item Leder for banKom
\end{liste}
}

\under{Advarsel og oppsigelse}{
Leder av en komite har rett til å si opp et medlem av sin egen komite. Oppsigelse skal kun finne sted i tilfeller der det blir ansett som høyst nødvendig for å beskytte komiteens samhold, initiativ, integritet eller profesjonalitet. Et komitemedlem har rett til å få én advarsel og en mulighet til å forbedre seg i forkant av en eventuell oppsigelse. Leder av komiteen plikter å konsultere leder av linjeforeningen i forkant av en eventuell advarsel eller oppsigelse. Det påpekes at leder av linjeforeningen har HMS-ansvar.
}

\section{Kontoret}

Kontorets retningslinjer defineres av trivselskomiteen. Det skal etterstrebes å holde kontoret åpent for sosialt samvær for medlemmer.

\newpage
\section{Ridderne av det Indre Lager}
Linjeforeningen har en egen orden for medlemmer som gjennom sitt arbeid har utmerket seg.

\subsection{Ridder av 1. grad}

Riddere av første grad er Æresmedlemmer av linjeforeningen.
Linjeforeningen har anledning til å utnevne æresmedlemmer. Æresmedlemskap går til personer som har gjort en eksepsjonell innsats for informatikkens sak, eller har gjort en eksepsjonell innsats for studentene ved informatikkstudiet. Æresmedlemmer har ikke noe noe krav om tidligere tilknytning til linjeforeningen for nominasjon.

\under{Medlemsskap}{
Utnevnelse skjer ved at Hovedstyret vurderer alle nominerte kandidater. Nominasjoner kan komme fra alle medlemmer av linjeforeningen. Kandidater utnevnes ved flertall i Hovedstyret.
Annonsering av opptak for riddere av 1. grad gjøres kun i sammenheng med jubileum.}

\subsection{Ridder av 2. grad}
Riddere av andre grad er Eldsterådet til linjeforeningen.
Eldsterådet er en gruppe bestående av særs erfarne medlemmer som har opparbeidet seg status og autoritet blant medlemmene i linjeforeningen. Eldsterådet skrives med stor ’E’ på samme måte som egennavn.

\under{Medlemsskap}{
Medlemmer av linjeforeningen blir medlem i Eldsterådet ved å oppfylle en eller flere av følgende krav:

\begin{liste}
	\item Er eller har vært leder av linjeforeningen.
	\item Er eller har vært et aktivt hovedstyremedlem i tre år (Seks semestre) eller mer.
	\item Har blitt utnevnt til Æresmedlem
\end{liste}

Man kan være medlem i Eldsterådet samtidig som man er medlem av Hovedstyret eller en av linjeforeningens komiteer.
Annonsering av opptak for riddere av 2. grad gjøres ved en egnet formell anledning der en stor mengde medlemmer er tilstede.}

\under{Formål}{
Det er Eldsterådets oppgave å avholde opptaksseremonien i forbindelse med linjeforeningsopptaket (ikke komiteopptak). Før opptaket skal sittende leder av linjeforeningen velge representanter fra Eldsterådet til å avholde opptaket sammen med lederen. Det bør være minst fem medlemmer av Eldsterådet til stede under opptaket. Leder er pliktig til å invitere Eldsterådet i god tid før linjeforeningsopptaket ettersommedlemmer av Eldsterådet gjerne bor langt vekk fra universitetet.}

\subsection{Ridder av 3. grad}

Riddere av tredje grad en personer som gjennom sin tid i linjeforeningen har vist stort engasjement, utført ekstraordinært arbeid eller har spesielt lang fartstid i frivillige verv tilknyttet informatikkstudiet.

\under{Medlemsskap}{
Utnevnelse skjer ved at Hovedstyret vurderer alle nominerte kandidater. Nominasjoner kan komme fra alle medlemmer av linjeforeningen. Kandidater utnevnes ved flertall i Hovedstyret.
Annonsering av opptak for riddere av 3. grad gjøres ved en egnet formell anledning der en stor mengde medlemmer er tilstede.}

\section{Mislighold av verv}
Om et komitemedlem eller en innehaver av linjeforeningsverv misligholder sine arbeidsoppgaver, kan ethvert medlem av linjeforeningen stille mistillitsforslag ovenfor vedkommende. Mistillitsforslaget skal leveres skriftlig til Hovedstyret, som skal behandle saken. Ved mistillitsforslag mot et hovedstyremedlem blir den anklagede suspendert inntil Hovedstyret har kommet med en avgjørelse. Mistillitsforslaget leses opp i Hovedstyret, deretter skal den anklagede få en mulighet til å forsvare seg før Hovedstyret diskuterer og avgjør saken uten den anklagede til stede. Dersom det stilles mistillitsforslag til flere styremedlemmer av gangen, skal disse behandles ved ekstraordinær generalforsamling. Det må ved innkalling til ekstraordinær generalforsamling fremmes mistillitsforsalg av minst 10 medlemmer.


\section{Vervvarighet}
Et verv i Linjeforeningen varer i tre år fra måneden man ble tatt opp. Dersom man ønsker å være et aktivt komitemedlem etter disse tre årene kan man søke til Hovedstyret om forlengelse av Online-vervet for ett år av gangen. Alle verv i Linjeforeningen teller på de tre vervårene, inkludert verv i Hovedstyret, men ekskludert verv i Seniorkomiteen

%--------------------------------------------
\newpage
